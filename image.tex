\documentclass[a4paper,11pt,onecolumn,twoside]{ctexart}
\usepackage[UTF8]{ctex}
\usepackage{hyperref}
\usepackage{amsmath,amssymb,amsfonts,amsthm,fancyhdr}
\usepackage{epsfig,graphicx,picins,picinpar,subfigure}
\usepackage{pstricks}
\usepackage{fancyvrb}
\usepackage[numbers,sort&compress]{natbib}
\usepackage{tikz}
\usepackage[linesnumbered, ruled, lined,boxed,commentsnumbered]{algorithm2e}[1]
\usepackage{caption}
\usepackage{graphicx}   % 导入图片
\usepackage{hyperref}
\usepackage{authblk}
\usepackage{fancyvrb}
\usepackage{amsmath}
\usepackage{amsthm}
\usepackage{amssymb}
\usepackage{tikz}
\usepackage[
    textwidth=14.8cm,
    textheight=21.5cm,
    top=0cm,
    left=0cm,
    right=0cm,
    footskip=0cm
]{geometry}
\setlength{\headsep}{0.5cm}
\setcounter{page}{1}
\pagestyle{fancy} \fancyhf{}
\newtheorem{mythrm}{定理}
\hypersetup{colorlinks=false,pdfborder={0 0 0},}

\begin{document}
许多几何问题可以使用复数简单地进行解决,第一次使用复数在几何方面取得重要
成果的人是高斯,他利用复数证明了正十七边形可以使用尺规作图画出。下面我们给出一个例子
以突显复数在解决几何难题的重要作用。\\

\textbf{问题} \quad
设$z_1,z_2,\cdots,z_n$是以原点为圆心的单位圆上的$n$个点$\left(n>2\right)$。如果$z_1,z_2,\cdots,z_n$
是正$n$边形的$n$个顶点,证明$z_1+z_2+\cdots+z_n=0$。\\

仔细分析这一问题,这里的$z_1+z_2+\cdots+z_n$是复平面的$n$个点相加。要想解决这一问题,
我们需要介绍几个基本的概念:

我们知道如果想要在复平面上确定一个复数,那么我们需要知道这一复数的实部和虚部。因此复数可以由
以下形式来表达:
\begin{equation}\label{复数的代数形式}
    z=x+iy   
\end{equation}
其中$i\triangleq\sqrt{-1}$。我们称$\left(\ref{复数的代数形式}\right)$式为复数的代数表示形式,$x$和$y$分别
被称为复数的实部和虚部。

类似地,复平面的复数可以使用类似极坐标的形式来表示,即当我们知道一个复数在复平面上相对$O$点的长度(一般称之为\textbf{模}),以及$O$点到该复数组成的射线相对于
实轴正半轴的角度(一般称之为\textbf{辐角}),那么我们就能在复平面上唯一确定一个复数。因此复数还可以由以下形式来表示:
\begin{equation}  \label{复数的指数形式}
    z=re^{i\theta}
\end{equation}
其中,$e$为自然指数,$r$表示复数的模,$\theta$表示复数的辐角,上式被称为复数的指数形式。有了这一形式,下面我们给出一个关于复数的重要定理:

\begin{mythrm}
在复数域有如下等式:
\begin{equation} \label{欧拉公式}
    e^{i\theta}=\cos\theta+i\sin\theta
\end{equation}
\end{mythrm}
\begin{proof}
    依次考虑函数$\cos x,\sin x,e^x$的泰勒展开:
    \begin{equation}\label{余弦函数的泰勒展开}
        \cos x=\sum_{n=0}^{+\infty}{\left(-1\right)}^n\frac{x^{2n}}{\left(2n\right)!}
    \end{equation}
    \begin{equation}\label{正弦函数的泰勒展开}
        \sin x=\sum_{n=0}^{+\infty}{\left(-1\right)}^n\frac{x^{2n-1}}{\left(2n-1\right)!}
    \end{equation}
    \begin{equation}\label{自然指数函数的泰勒展开}
        e^x=\sum_{n=0}^{+\infty}\frac{x^n}{n!} 
    \end{equation}
    我们将$x=i\theta$带入$\left(\ref{自然指数函数的泰勒展开}\right)$式可以得到:
    \begin{equation}
        e^{i\theta}=\overbrace{\sum_{n=0}^{+\infty}{\left(-1\right)}^n\frac{\theta^{2n}}{\left(2n\right)!}}^{\cos\theta}+i\overbrace{\sum_{n=0}^{+\infty}{\left(-1\right)}^n\frac{\theta^{2n-1}}{\left(2n-1\right)!}}^{\sin\theta}
    \end{equation}
    由此我们得到:
    \begin{equation*}
        e^{i\theta}=\cos\theta+i\sin\theta
    \end{equation*}
    即为$\left(\ref{欧拉公式}\right)$,证毕。
\end{proof}
第一个发现该等式且给出证明的是欧拉,因此该公式也被称之为\textbf{欧拉公式}。限于篇幅,我们给出了这一定理的一个不太严谨的证明。这一定理将复数的代数形式和指数形式相结合,得到了
一个新的复数表示形式:
\begin{equation} \label{复数的三角形式}
    z=r\left(\cos\theta+i\sin\theta\right)
\end{equation}
$\left(\ref{复数的三角形式}\right)$式被称为复数的三角形式。

在实际应用中,我们可以根据实际需要来灵活运用和转化三种不同的复数表达形式,以此来减轻工作量。\\
\begin{figure}[h]
    \begin{center}
        \begin{tikzpicture}
            \draw[->](0,0)--(5,0);
            \draw[->](2.5,-2.5)--(2.5,2.5);
            \draw[->](2.5,0)--(4.2,1.5);
            \draw(2.8,0) arc (0:40:0.3);
            \node[left]at(2.5,-0.3){$O$};
            \node[right]at(5,0){$x$};
            \node[above]at(2.5,2.5){$y$};
            \node[above]at(3.1,0){$\theta$};
            \node[centered]at(3.5,1.1){$r$};
            \node[centered]at(4.5,1.8){$z$};
        \end{tikzpicture}
    \end{center}
    \caption{复数$z$的辐角和模长}
\end{figure}

在引入了关于复数的表达形式以及欧拉公式之后,我们给出一个解决问题的重要定理:\\

\begin{mythrm} \label{定理二}
    对于一元方程$z^n=\zeta,\zeta=re^{i\theta}\in \mathbb{C}\left(\zeta\neq 0,n>0\right) $,其在复数域上一定存在n个互不相等的根$z_1,z_2,\cdots,z_n$,且
    这些根满足:
    \begin{equation} \label{一元方程的根}
        z_k=\sqrt[n]{r}e^{i\frac{\theta +2k\pi}{n}} \quad k=0,1,2,\cdots,n-1
    \end{equation}
\end{mythrm}
\begin{proof}
    由$\zeta=re^{i\theta}$可知,复数$\zeta$的模长为$r$,辐角为$\theta$。
    对于任一复数$z=r_0e^{i\theta_0}$,由复数的指数表达进行计算可知:
    \begin{equation*}
        z^n=r_0^ne^{in\theta_0}
    \end{equation*}
    此时,$z^n$的辐角变为原来的$n$倍,模长为原来的$n$次方。
    由此可知,若对一个复数进行$n$次方运算,其模长变成了$r$,辐角变成了$\theta$,则该复数的模长
    应该为$\sqrt[n]{r}$,其辐角应该为$\frac{\theta +2k\pi}{n}$。这样的复数有无穷多个,但考虑
    到由于$k$的选取会导致一部分复数在复平面上的位置完全重复,因此这些复数可以分成$n$类,我们可以从这些类
    中选取较为简单的一些,即为$k=0,1,2,\cdots,n-1$的情况,因此结论成立。
\end{proof}
如果当$n>2$时观察$\left(\ref{一元方程的根}\right)$中$n$个根在复平面上的位置,很容易注意到以这$n$个根为顶点可以组成一个正$n$边形。
由此,我们来考虑最开始的问题:
\begin{proof}
    由于$z_1,z_2,\cdots,z_n$为正$n$边形的$n$个顶点,且这$n$个点分布在复平面的单位圆周上,不失一般性地,我们令:
    \begin{equation}
    z_k=e^\frac{\theta+2k\pi}{n} \quad k=1,2,\cdots,n
    \end{equation}
    显然,这$n$个复数为多项式$z^n=e^{i\theta}$的$n$个互异的根。因此
    \begin{equation}  \label{因式分解}
    z^n-e^{i\theta}=\prod_{k=1}^{n}\left(z-z_k\right)
    \end{equation}
    分析$\left(\ref{因式分解}\right)$式,左式是一个$n$次多项式,且该多项式除了$n$次项以及$0$次项之外,其他项的系数都为$0$。
    右式也一样为一个$n$次多项式,我们只考察其$n-1$次方项,很容易得到右式的$n-1$次方项为:
    \begin{equation}
        a_{n-1}=-\sum_{k=1}^{n}z_k
    \end{equation}
    由于左右两个多项式相等,则右式的$n-1$次项也应该为$0$,即
    \begin{equation}
        \sum_{k=1}^{n}z_k=0
    \end{equation}
\end{proof}
\end{document}